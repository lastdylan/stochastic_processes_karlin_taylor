%%=====================================================================================
%%
%%       Filename:  one.tex
%%
%%    Description:  
%%
%%        Version:  1.0
%%        Created:  06/30/2019
%%       Revision:  none
%%
%%         Author:  Last Feremenga, Ph. D.
%%   Organization:  
%%      Copyright:  Copyright (c) 2019, YOUR NAME
%%
%%          Notes:  
%%
%%=====================================================================================

\begin{center}
{\LARGE \bf Chapter 2} \\
{\LARGE Elements of Stochastic Processes}\\
\end{center}

\section{Markov Chains}
\par The definition of a Markov chain is quite clear from Section 1. Of usual interest are {\it discrete
time} Markov chains, where the state space is countable and finite. These states are normally
labelled by nonnegative integers, since there can be found a one-to-one mapping between the states
and a subset of integers. The time at which the chain is evaluated is also labelled by a nonnegative
integer. So, a chain that is in state $i$ at time $n$ is represented by
\begin{equation}
X_n = i
\end{equation}

\par The probability of $X_{n+1} = j$ given that $X_n = i$, known as the transition probability, is then
\begin{equation}
P_{ij}^{n,n+1} = \pr{X_{n+1}=j|X_n=i}.
\end{equation}
Most chains of interest are {\it stationery}, indicating that the transition probabilities are
independent of $n$. So, the transition probabilities are represented simply as $P_{ij}^{n, n+1}$.

\par A Markov chain is fully defined by specifying all the transition probabilities. This can be done
by defining a {\it transition probability matrix}, $\vec{P} = \norm{P_{ij}}$.

\subsection{Example Markov Chains}
\par Refer to Elementary Problem 1a. for a full problem description. In brief, we are to define
$\vec{P} = \norm{P_{jk}}$ for a sequence of coin tosses (with probability of heads $p$)
where the state of the process is the number of heads minus the number of tails at time $n$. So,
it would suffice to find

\begin{equation}
P_{jk} = \pr{H^{n+1}-T^{n+1}=k|H^n-T^n=j}.
\end{equation}

We notice that $H^{n+1} = H^n + h$ and $T^{n+1} = T^n + t$, where $h=1, t=0$ if we toss heads
at $n+1$ time otherwise $h=0, t=1$. So,
\begin{equation}
P_jk = \pr{H^n+h-T^n-t=k} = \pr{h-t=k-j}
\end{equation}
When $k-j=1$ we'd have tossed heads; we get that with probability $p$. When $k-j=-1$ we'd have
tossed tails; we get that with probability $1-p$. So
\begin{equation}
P_{jk} =
    \begin{cases}
    p, & k=j+1 \\
    1-p, & k=j-1 \\
    0, & \text{otherwise}
    \end{cases}
\end{equation}

\par Refer to Elementary Problem 1b. for a full problem description. In brief, we are to find a
$\vec{P}$ for a system of two urns that at $n=0$ contain $N$ black and white balls each, respectively.
At each time step, balls are randomly picked from each urn and swapped between the urns. The state of
this system is the number of white balls in the first urn. We label these urns $\beta,\omega$
respectively. Now
\begin{equation}
P_{jk} = \pr{W^{n+1}_\beta = k|W^n_\beta = j}
\end{equation}
where $W^n_\beta$ is the variable for the number of white balls in the $\beta$ urn at time $n$.
Given that $W^n_\beta = j$, we know that $W^n_\omega = N-j$. For $k=j+1$ we need to pick a white
ball from $\omega$ and a black ball from $\beta$. For $k=j$ we need to pick either both black balls
or white balls. For $k=j-1$ we need to pick a black ball from $\omega$ and a white ball from $\beta$.
These cases clearly give
\begin{equation}
P_{jk} =
    \begin{cases}
    \frac{(N-j)^2}{N^2}, & k=j+1 \\
    \frac{i^2}{N^2}, & k=j-1 \\
    \frac{2j(N-j)}{N^2}, & k=j \\
    0, & \text{otherwise}
    \end{cases}
\end{equation}
We can convince ourselves by checking that these transition probabilities sum up to 1.
